\section{研究计划进度安排及预期目标}

\subsection{进度安排}
\begin{enumerate}
    \item 2021.10.10前:选定课题;
    \item 2021.10.11-11.22:阅读国内外文献(情绪心理学、情绪生理唤醒等相关研究、综述和现有数据库),完成文献综述与英文文献翻译,同时熟悉心理学实验流程和规范;
    \item 2021.11.22-2022.1.4:完成开题报告;参与基于外周生理信号的情绪识别实验设计(包括情绪刺激材料选取的心理学实验和测量情绪生理唤醒的实验),完成生理数据的采集;
    \item 2022.1.1-2022.3.6:进行数据预处理、特征参数计算及统计学分析,提取识别恐惧、愉悦两种情绪的有效参数并构建情绪特征集;
    \item 2022.3.7-2022.3.11:完成中期检查;
    \item 2022.3.12-2021.5.7:采集更多的实验数据,并进行进一步的数据处理和分析,优化情绪识别模型;
    \item 2022.5.8前:完成毕业论文并上传;
    \item 2022.5.9-2022.5.22: 完成毕业论文盲审评阅;
    \item 2022.5.23-2022.5.30: 完成毕业论文修改,准备答辩;
    \item 2022.6.1-2022.6.5: 完成毕业论文答辩。
\end{enumerate}

\subsection{预期目标}
本毕业设计的预期目标主要有三点:
\begin{enumerate}
    \item 完成情绪唤起实验的方案设计和多种外周生理信号的采集及预处理;
    \item 分析采集得到的外周生理信号在多个特征空间维度(时域、频域、非线性)的生理特征,进一步运用统计学方法选取识别恐惧和愉悦两种情绪的有效特征,并探索有效特征与情绪状态之间的内在关联;
    \item 研究机器学习方法,构建恐惧平静二分类和愉悦平静二分类模型并进行优化,实现对恐惧和愉悦两种情绪较为准确地识别。
\end{enumerate}
