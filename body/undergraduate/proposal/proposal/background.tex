\section{问题提出的背景}

\subsection{背景介绍}
情绪是人类生活中极其重要的心理活动,表现为人对客观事物的态度体验及相应的行为反应\cite{Mengzhaolan2005}。
医学和心理学领域的大量研究和临床证据表明\cite{Moskowitz2008,Dongyan2012,Ong2006,Caprara2020,Trivedi2020,Trivedi2020},情绪的正负向及其长期的积累效应在很大程度上影响着疾病的产生、诊断及治疗过程。积极情绪可以间接作用于机体生理系统,提高疾病预防或康复的能力,而消极情绪则效果相反。
此外,情绪还会影响工作效率与质量,影响个体个性发展、自我认识及评价,影响人们的解释、判断、决策、推理及风险认知等高层次的认知过程。
因此,有关情绪识别、评估及干预、治疗的研究意义重大,其应用前景逐渐在临床治疗、人工智能、司法、教育、认知负荷等多个领域展现。

虽然目前心理学界对情绪的定义尚未形成统一的观点,但其研究取向中比较一致的地方在于,情绪往往都伴随一定的主观体验、外部表现和生理唤醒。主观体验即个体对不同情绪和情感状态的自我感受;
外部表现即表情,包括面部表情、姿态表情和语调表情;生理唤醒即情绪产生的生理反应和变化,它与广泛的神经系统有关。基于以上研究者普遍认可的情绪三成分,目前情绪的测量及识别方法主要分为三类:

第一类方法是通过量表的形式,对情绪的主观体验进行评价或对情绪的基本维度进行测量。这种方法在心理学研究中非常常见,具有成熟的理论体系且使用简便,但缺点是效率很低,
且往往具有滞后性,不利于在实际环境中大规模应用。同时,问卷受主观影响很大,会导致结果在客观性上有所欠缺。

第二类方法是通过面部表情、身体姿态和语音信息进行情绪识别。这种方法相对简便有效,适合在日常生活中推广使用,且已经有大量的相关研究及实践检验,
但缺点是容易受到多种因素(个体因素、社会文化及情境因素、刺激因素和疾病因素等)的影响,导致个体表情被不同程度地修饰、夸大、压抑或掩饰,使得结果在一致性上有所欠缺。

第三类方法是通过测量外周自主神经系统的反应、大脑脑区的活动变化或体内一些神经化学物质的改变来考察各项生理指标变化与情绪之间的对应关系。这种方法不受人的主观意愿控制,因此,
可以通过检测这些变化信息来识别情绪状态。研究者常用的、对不同基本情绪反应敏感的生理信号主要包括自主神经反应的五大类测量指标(心血管、皮肤电、呼吸和肠胃测量)和瞳孔、眨眼以及测量中枢神经反应的脑电图、脑磁图等。
其中有关脑电信号的研究最多,且已有研究结果证实不同的情绪状态能明显地引起脑电信号不同的变化。例如,愉悦情绪下脑电信号α波和γ波的功率谱密度以及微分熵通常大于恐惧,β波则没有明显差异,而差分不对称和理性不对称特征则在恐惧情绪下增强,愉悦情绪下减弱\cite{LiuShuang2015}。
脑电信号在情绪状态改变时变化明显且能有效检测,但其采集环境和设备要求较为严格,不利于长时期和随时采集,也不利于情绪识别装置的小型化和便携化。同时,脑电采集过程可能还会给被试造成不适感,引起被试的情绪变化并干扰情绪识别过程。

相较于脑电信号而言,外周生理信号更易采集和分析,采集时对被被试情绪干扰也更少,更适用于长期和未来应用于日常生活中的情绪监测。
并且融合多种生理信号的信息能很好地提取出不同情绪状态的固有特性,有利于分析它们之间的模式差异。
另一方面,大量研究也表明外周生理信号同样对情绪变化敏感,不同情绪在皮肤、心率、血压、指温、心率变异性等生理指标上存在一定的差异\cite{Kreibig2010},可以传达有关情绪状态的信息。
因此,基于外周生理信号的情绪识别研究具有非常重要的意义。

\subsection{基于外周生理信号进行情绪识别的研究现状}
继James和Lange等情绪理论家\cite{James1884,Malatesta1987}提出相关情绪外周理论之后,很多研究人员开始重视外周自主神经反应在情绪产生中的作用,他们尝试通过多种实验手段找到人类基本情绪所对应的外周生理反应模式,
也确实发现人类所体验到的不同情绪在皮肤、心率、血压、指温、心率变异性等生理指标上存在一定的差异\cite{Kreibig2010},为后续研究建立基础。
1997年,Picard\cite{Picard1997}首次提出“情感计算”的概念,并在2001年带领团队\cite{Picard2001}收集了一位女演员在数周内每天表达八种情绪状态时的肌电、血容量脉冲、呼吸和皮肤电信号,提取信号均值、标准差等40多个特征构建高维情感特征空间后,采用序列浮动前向选择法和Fisher投影法处理特征,并采用K近邻算法进行分类,获得了81\%的识别准确率,证明了从生理信号中提取特征进行情绪识别研究的可行性。之后,利用生理信号进行情绪识别的研究逐渐发展起来。

实验范式方面,研究中主要采用电影片段情绪诱发、文字/图片情绪诱发、录音/音乐情绪诱发、自传式回忆/想象情绪诱发和具身情境性情绪诱发等典型的情绪诱发方式。
例如,李建平等\cite{LiJianping2006}选用《我的兄弟姐妹》、《午夜凶铃》等六段电影片段诱发92名被试五种基本情绪,发现悲伤、愤怒、恐惧及中性片段导致收缩压升高,厌恶、愤怒、恐惧、快乐和中性片段导致呼吸频率加快,悲伤、恐惧和中性片段导致HRV高频功率降低。
Gomez等\cite{Gomez2004}采用不同正负效价和唤醒度的情绪图片来诱发情绪,发现随着图片效价的增加,被试吸气时间延长,平均吸气流量减少,胸式呼吸增加;随着图片唤醒度的增加,被试吸气时间和总呼吸时间缩短,平均吸气流量、每分钟通气量、胸式呼吸和皮肤电活动增加。
Egloff等\cite{Egloff2002}通过让被试进行演讲的方式诱发其焦虑情绪,发现被试的指端脉搏容积和呼吸频率下降,心率和血压则显著增加。
同时,研究人员也尝试建立标准化的情绪刺激材料数据库,如
美国国立精神卫生研究所建立的标准化国际情绪图片库(IAPS)和英语情感词/短文系统(ANEW/ANET)中国学者\cite{BaiLu2005}建立的本土化中国情绪图片/词库 (CAPS),
IMH建立的国际情感数码声音系统(IADS)和刘涛生等中国学者\cite{LiuTaosheng2006}建立的本土化的中国情感数码声音系统(CADS)等等,这些数据库为统一情绪刺激材料选取范围和标准进行了有效探索。

分析方法方面,研究中主要采用传统的机器学习方法对信号进行特征的提取、选择与融合处理以构建、优化情绪识别模型。
常用的外周生理信号特征\cite{Bota2019}包括时域内的最大/小值、均值/中位数、标准差/方差、平均绝对偏差、过零率、线性回归和积分等,频域内的总能量、光谱质心/扩展/偏度/峰度/斜率/衰减/变化等,非线性中的谱熵、小波熵、样本熵、近似熵、关联维数和Lempel-Ziv复杂度等。
例如,Rubin等\cite{Rubin2016}在恐惧状态的识别中将ECG信号的传统时频域特征和非线性特征相结合,显著提升了模型准确度。
Myroniv等\cite{Myroniv2017}基于RT、DT、NB、KNN、SVM和MPNN模型对HR、GSR和SKT信号进行分类,其中KNN模型的平均准确率达到了最高的97.78\%。
Birjandtalab等\cite{Birjandtalab2016}基于GMM模型对EDA、HR和${\rm SpO}_{\rm 2}$信号进行分类,针对放松、生理紧张、情绪紧张和认知紧张这4种状态均得到了大于84\%的准确率。
大量的生理特征被用于情绪识别领域中,为寻找情绪变化的可靠指标提供了基础。

除传统机器学习方法之外,不依赖于特征的深度学习方法(DL)也逐渐应用到情绪识别的研究当中。
杨一龙等\cite{YangYilong2018}使用一种结合CNN和RNN的混合神经网络,通过学习原始EEG数据的组成时空表征来区分情绪状态,最终将情绪识别准确率提高了约32\%,在效价分类和唤醒分类上的平均准确率分别为90.80\%和91.03\%。
陈景霞等\cite{ChenJX2019}提出一种基于CNN模型的使用EEG特征进行情绪分类的模型,并对比了该模型与决策树、支持向量机、线性判别分析、贝叶斯线性判别分析等在DEAP标准情绪数据集上的识别效果,发现深度学习模型在结合时域与频域的特征集上获得了最好的识别效果。利用深度学习进行情绪识别的有效性在大量研究中得到证明,
其识别效果与准确率甚至优于部分传统的机器学习模型,但其缺陷在于解释性很差,不能表征生理信号和情绪之间的对应关系,同时也需要大量的数据和较高的计算成本。

综上所述,目前已经有许多基于生理信号进行情绪识别的研究。随着研究的深入,诱发的情绪种类逐渐复杂,使用的生理信号变得多样化,提取的特征参数更为有效,识别模型的性能也逐步提升。在完善情绪诱发方案,构建标准情绪数据库,
推动情绪识别系统开发等多个方面也都获得了丰富的成果。但仍存在着以下三个问题:

(1) 现有研究较为依赖脑电信号,但脑电信号的采集过程(包括操作、设备等)相对复杂和困难,会使得研究难以适用于长时间的情绪监测以及情绪识别进入日常生活的应用趋势;同时,采集过程中的不适感也可能会对被试情绪产生更多影响,容易造成一些情绪识别错误。

(2) 现有研究所提取的情绪特征较为单一,多为统计、时域、频域和非线性特征中的一种或几种,没有全面地从多个维度提取生理信号中的情绪信息,导致最后模型识别准确率降低。

(3) 现有研究多着眼于分类准确率的提高,重点关注生理信号的选择、特征的提取、分类器的优化等方面内容,缺乏对生理信号固有属性的变化与不同情绪之间关系的研究。目前对情绪和生理特征之间的特异性关联缺乏全面的认识,情绪识别的研究结果缺乏成熟的理论进行解释和分析。

\subsection{本研究的意义和目的}
在常见的负性情绪中,恐惧是一种影响较深的高唤醒度状态\cite{James1884},产生于自身感知到可能受到威胁的情况,根本目的在于保护机体免受外界伤害。但长期处于恐惧状
态会加深人们对负面信息的记忆,形成条件化恐惧,导致精神压力过大、工作效率下降、免疫力降低,甚至会产生创伤后应激障碍,对个体的身心健康产生严重
伤害。而在面对应激事件而产生消极情绪和失控行为的情况下,愉悦等积极情绪则可以有效降低机体的消极反应,抵消部分负面影响。因此对恐惧和愉悦情绪的研究对于稳定情绪状态和改善生活质量有很大的实际意义。
论文希望融合多种外周生理信号,提取多个维度的特征参数对恐惧和愉悦情绪进行识别,更加全面和深入地研究相关外周生理特征与两种基本情绪之间的关联,以达到更好的识别效果。
同时,研究聚焦外周生理信号,有利于后续情绪识别研究在实验设备小型化、便携化上的进一步拓展,有利于后续进一步研究情绪识别在日常生活中的应用。
