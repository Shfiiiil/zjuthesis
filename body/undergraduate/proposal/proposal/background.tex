\section{问题提出的背景}

\par 正文格式与具体要求\cite{zjuthesisrules}

\subsection{背景介绍}
情绪是人类生活中极其重要的心理活动,表现为人对客观事物的态度体验及相应的行为反应。
医学和心理学领域的大量研究和临床证据表明,情绪的正负向及其长期的积累效应在很大程度上影响着疾病的产生、诊断及治疗过程。积极情绪可以间接作用于机体生理系统,提高疾病预防或康复的能力,而消极情绪则效果相反。
此外,情绪还会影响工作效率与质量,影响个体个性发展、自我认识及评价,影响人们的解释、判断、决策、推理及风险认知等高层次的认知过程。
因此,有关情绪识别、评估及干预、治疗的研究意义重大,其应用前景逐渐在临床治疗、人工智能、司法、教育、认知负荷等多个领域展现。

虽然目前心理学界对情绪的定义尚未形成统一的观点,但其研究取向中比较一致的地方在于,情绪往往都伴随一定的主观体验、外部表现和生理唤醒。主观体验即个体对不同情绪和情感状态的自我感受;
外部表现即表情,包括面部表情、姿态表情和语调表情;生理唤醒即情绪产生的生理反应和变化,它与广泛的神经系统有关。基于以上研究者普遍认可的情绪三成分,目前情绪的测量及识别方法主要分为三类:

第一类方法是通过量表的形式,对情绪的主观体验进行评价或对情绪的基本维度进行测量。这种方法在心理学研究中非常常见,具有成熟的理论体系且使用简便,但缺点是效率很低,
且往往具有滞后性,不利于在实际环境中大规模应用。同时,问卷受主观影响很大,会导致结果在客观性上有所欠缺。

第二类方法是通过面部表情、身体姿态和语音信息,对被试的情绪进行识别。这种方法较为简便有效,适合在日常生活中大量使用,且已经有大量的相关研究及实践检验,
但缺点是容易受到多种因素(个体因素、社会文化及情境因素、刺激因素、疾病因素等)的影响,导致个体表情被不同程度地修饰、夸大、压抑或掩饰,使得结果在一致性上有所欠缺。

第三类方法是通过测量外周自主神经系统的反应、大脑脑区的活动变化或体内一些神经化学物质的改变来考察各项生理指标变化与情绪之间的对应关系。这种方法不受人的主观意愿控制,因此,
可以通过检测这些变化信息来识别情绪状态。研究者常用的、对不同基本情绪反应敏感的生理信号主要包括自主神经反应的五大类测量指标(心血管、皮肤电、呼吸和肠胃测量)和瞳孔、眨眼以及测量中枢神经反应的脑电图、脑磁图等。
其中有关脑电信号的研究最多,且已有研究结果证实不同的情绪状态能明显地引起脑电信号不同的变化。例如,快乐时脑电信号 α 波和 γ 波的功率谱密度大于恐惧,β 波则没有明显差异;在积极情绪中的微分熵通常高于消极情绪;而差分不对称和理性不对称特征,在恐惧情绪下增强,快乐状态下减弱。脑电信号在不同的情绪状态中变化明显,是一种使用广泛且有效的信号。但脑电信号的采集有着较为严格的环境和设备要求,不便于随时或者长时间地进行采集,也不利于情绪识别装置的小型化和便携化。脑电采集时还会给受试者造成不适感,引起被试的情绪变化并干扰情绪识别过程。

外周生理信号情绪识别的优势。。。因此,基于外周生理信号的情绪识别研究具有非常重要的意义。目前


\subsection{基于外周生理信号进行情绪识别的研究现状}

\subsection{本研究的意义和目的}
