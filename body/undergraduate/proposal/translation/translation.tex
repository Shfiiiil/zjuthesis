\cleardoublepage
\chapter{外文翻译}

\sectionnonum{K-EmoCon:用于自然对话中连续情感识别的多模式传感器数据集}
\textbf{摘要:}随着低成本移动传感器的普及,社交互动中情感识别的潜在应用前景逐渐显现出来,但同时也面临着缺乏自然情感互动数
据的挑战。大多数现有的情绪数据集是在有限环境中收集的,所以它们并不支持研究在自然状态下产生的特殊情绪。
因此,研究社会互动背景下的情绪需要一个全新的数据集,而K-EmoCon就是这样一个对自然对话中连续情绪
进行全面注释的多模态数据集。它运用现成设备,在16场(每场约10分钟)有关社会问题的配对辩论中进行多模态测量
,测量值包括视听记录、脑电图和外周生理信号等。与之前数据集不同的是,它采用了来自三个可用视角的情感注释:
自我、辩论同伴和外部观察员。评审员在观看辩论录像时,每隔5秒对情绪表现(包括唤起效价和18种额外的分类情绪)
进行注释,由此产生的K-EmoCon是适用于社会互动中情绪多视角评估的第一个公开可用的情绪数据集。

% \textbf{关键词:} 1 2 3
\section{背景和总结}
情绪识别研究旨在赋予计算机识别情绪的能力,它是创造能够理解甚至表达情感的机器的基础,
而这一套识别、理解和表达情绪的技能构成了我们所谓的情商。研究表明,情商对于个体在社会中的导向是非常必要的
,因为它能够引导一个人判断事物是否可取,并相应地规范自己和他人的行为。
那么为什么机器需要情感技能?随着机器学习和人工智能的进步,在社会的各个领域,
包括那些需要专业知识的领域,如医疗预后/诊断或汽车驾驶,从人类到机器的过渡已经显而易见。细分领域的人工智能系统似乎不可避免地在各自的领域超越了人类专家,这一点已经被AlphaGo在围棋比赛中超越人类棋王的出色表现所证明。
不过,尽管人工智能拥有超人的能力,但并非所有的人工智能都会与人类竞争。相反,许多人工智能系统会与我们合作或为我们服务,而情商在这种人机交互系统中发挥了至关重要的作用。想象一下,当用户回家时,一个智能音箱会愉快地打招呼。但当用户度过了不愉快的一天时,音箱又该如何问候?忽视用户情绪状态的音箱可能会加重用户的负担,但意识到用户心情的智能音箱可以保持沉默以避免麻烦。同样,对于设计用于复杂任务的人工智能系统来说,情商也至关重要。例如,在自动驾驶和人工驾驶混合的道路上,准确识别人类驾驶员的情绪能够使自动驾驶汽车更好地判断人类驾驶员的意图,从而带来更高的安全性。
现在,机器要想拥有情商,就必须首先学会识别情绪,而学习的先决条件就是数据。然而,在情绪数据的获取方面还存在一些挑战。虽然情绪是普遍存在的,但要准确地衡量它们是困难的。情绪通常被认为是通过面部表情表达的心理状态,它有不同的类别。但后续研究却产生了完全相反的结论,面部表情不是清晰的,而是复合的、相对的和具有误导性的。最近一篇对相关科学证据的综述也反对这一普遍观点,认为面部表情缺乏可靠性、专一性和概括性,过去对情境依赖性和情绪的个体可变性的研究也是如此。
情绪这种内在的隐蔽性使得现有的许多情绪数据集不适合在自然状态下研究情绪。大多数情绪数据集由静态环境(即实验室)中挑选的特定刺激所诱发的情绪组成。这种方法使得实验者能够完全控制数据的收集,并进一步对特定的情感行为进行评估以及通过先进技术(如神经影像学)获取精密细粒度的数据。然而,实验室生成的数据对现实场景的概括性很差,因为它们经常包含典型情绪的强烈表达,而这在现实世界中是很少能被观察到的,并且只是从一小部分人群中所获得。
另一种利用媒体内容和众包的方法,弥补了传统方法的缺陷。丰富的网络内容,如电视节目和电影,使得研究人员能够有效收集各种背景下的丰富情感数据;众包则在作为另一个数据源的同时,进一步支持完成成本较低的数据注释。这种类型的数据集在样本量和受试者多样性方面具备优势,但存在普遍性方面的缺陷。基于媒体内容的数据集,其情感展示通常由专业演员通过假想情境而产生,而这种情感描绘与自发的情感表达是否相似以及有多大程度相似仍然存在争议。同时,这类数据集也无法获取生理信号,而生理信号恰恰被认为是检测微弱情绪状态变化的关键信息。
为了弥补在识别自然情绪方面数据集的缺陷,我们引入了K-EmoCon,这是一个从32名被试(两两成组进行有关社会问题的辩论)中采集的多模式数据集。它包括用三个现成可穿戴设备收集的生理传感器数据、辩论期间参与者的视听录像以及连续情绪注释。据我们所知,K-EmoCon也是第一个从所有可能角度(被试主体、辩论伙伴和外部观察者)进行情感注释的数据集。

\subsection{方法}
\subsubsection{预期的用途}
