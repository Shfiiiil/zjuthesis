\cleardoublepage
% \begin{spacing}{1.5} 
%     \bibliography{refs}   % 参考文献
%   \end{spacing} 

\setcounter{page}{1}
\section{绪论}

\subsection{背景}

情绪是人类生活中极为重要的心理活动,表现为人对客观事物的态度体验及相应的行为反应\cite{Mengzhaolan2005}。
医学和心理学领域的大量研究和临床证据表明\cite{Moskowitz2008,Dongyan2012,Ong2006,Caprara2020,Trivedi2020,Trivedi2020},情绪的正负向及其长期的积累效应在很大程度上影响着疾病的产生、诊断及治疗过程。积极情绪可以间接作用于机体生理系统,提高疾病预防或康复的能力,而消极情绪则效果相反。
此外,情绪还会影响工作效率与质量,影响个体个性发展、自我认识及评价,影响人们的解释、判断、决策、推理及风险认知等高层次的认知过程。
因此,有关情绪识别、评估及干预、治疗的研究意义重大,其应用前景逐渐在临床治疗、人工智能、司法、教育、认知负荷等多个领域展现。

目前心理学界主要有两种主流的情绪模型,离散情绪模型和维度情绪模型。离散情绪模型即情绪分类取向,其研究者认为情绪主要由几种相对独立的基本情绪以及由基本情绪结合形成的多种复合情绪所构成\cite{Mengzhaolan2005}。
如Ekman将情绪描述为离散的、可测量的和与生理学相关的,存在快乐、恐惧、悲伤、愤怒、惊讶与厌恶六种基本情绪\cite{Ekman1971,Ekman1992},该基本情绪分类学说也成为目前影响力最大的情绪分类方法之一。
维度情绪模型即情绪维度取向,认为情绪是高度相关的连续体,无法区分为独立的基本情绪,并且同类情绪在其基本维度上都高度相关,但有关基本维度的数量和类型还存在一定争议\cite{Mengzhaolan2005}。
Mehrabian和Russell提出了情绪状态的三维度模型,即效价唤醒度支配度(PAD)\cite{Mehrabian1974},其中效价的范围从消极到积极,表示情绪从不愉快到愉快;唤醒度的范围从被动到主动,表示情绪的激烈程度,支配度的范围从受控到主控,表征个体在某种情绪中的控制能力。
各类情绪根据这三个维度的不同特性,分布在情绪空间的不同象限中。维度情绪模型将不同情绪之间的共性整合并以此搭建维度空间体系,通过连续维度的方式对情绪进行定量分析,摆脱了离散情绪模型在复杂情绪表征上的局限,也更适用于研究及实际应用。

虽然目前心理学界对情绪的定义尚未形成统一的观点,但其研究取向中比较一致的地方在于,情绪往往都伴随一定的主观体验、外部表现和生理唤醒\cite{Mengzhaolan2005}。主观体验即个体对不同情绪和情感状态的自我感受;
外部表现即表情,包括面部表情、姿态表情和语调表情;生理唤醒即情绪产生的生理反应和变化,它与广泛的神经系统有关\cite{Mengzhaolan2005}。基于以上研究者普遍认可的情绪三成分,目前情绪的测量及识别方法主要分为三类:

\raggedbottom
第一类方法是通过量表的形式,对情绪的主观体验进行评价或对情绪的基本维度进行测量。这种方法在心理学领域中发展相对成熟和规范,
其中应用最广泛的是自我评估量表(SAM),使用非常简便,但效率很低且具有滞后性,不利于在日常生活中推广使用\cite{陈沙利2021}。
同时,问卷受主观影响很大,会导致结果在客观性上有所欠缺。

第二类方法是通过面部表情、身体姿态和语音信息进行情绪识别。这种方法目前已经有大量的相关研究及实践检验,操作相对简便有效,适合在日常生活中推广使用,
但缺点是容易受到多种因素(个体因素、社会文化及情境因素、刺激因素和疾病因素等)的影响,导致个体表情被不同程度地修饰、夸大、压抑或掩饰,使得结果在一致性上有所欠缺。

第三类方法是通过测量外周自主神经系统的反应、大脑脑区的活动变化或体内一些神经化学物质的改变来考察各项生理指标变化与情绪之间的对应关系。这种方法不受主观意愿控制且生理反应具有一定的相似性,结果的客观性和一致性较好。
研究者常用的、对不同基本情绪反应敏感的生理信号主要包括自主神经反应的五大类测量指标(心血管、皮肤电、呼吸和肠胃测量)和瞳孔、眨眼以及测量中枢神经反应的脑电图、脑磁图等。
其中脑电信号相关的研究最多,例如,刘爽等发现积极情绪下的脑电信号α、γ波的功率谱密度以及微分熵通常大于消极情绪,β波则没有明显差异,
同时愉悦情绪下的不对称指数明显减弱而恐惧情绪下则明显增强\cite{LiuShuang2015}。
虽然已有大量研究结果证实脑电信号在情绪状态改变时变化明显且能有效检测,但其采集环境和设备要求较为严格,不利于随时或长时程采集,也不利于情绪识别设备的便携化。

相较于脑电信号而言,外周生理信号更易采集和分析,采集时对被被试情绪干扰也更少,更适用于长期和未来应用于日常生活中的情绪监测。
同时,外周生理信号可以进行多信号融合分析,从多个角度提取出与情绪状态具有相关性的生理特征,有利于分析不同情绪之间的模式差异。
另一方面,大量研究也证实基于外周生理信号进行情绪识别的可行性,不同情绪在皮肤、心率、血压、指温、心率变异性等生理指标上存在一定的差异\cite{Kreibig2010},
可以有效表征不同的情绪状态。因此基于外周生理信号进行情绪识别研究意义重大。

\clearpage
\subsection{基于外周生理信号进行情绪识别的研究现状}
继James和Lange等情绪理论家\cite{James1884,Malatesta1987}提出相关情绪外周理论之后,很多研究人员开始重视外周自主神经反应在情绪产生中的作用,他们尝试通过多种实验手段找到人类基本情绪所对应的外周生理反应模式,
也确实发现人类所体验到的不同情绪在皮肤、心率、血压、指温、心率变异性等生理指标上存在一定的差异\cite{Kreibig2010},为后续研究建立基础。
1997年,Picard\cite{Picard1997}首次提出“情感计算”的概念,并在2001年完成了基于肌电、皮肤电、呼吸和血容量脉冲等信号的情绪识别研究,在数周内连续采集一位女演员每天表达八种情绪状态时的生理信号并提取时域、频域等范围内40多个特征,使用序列浮动前向选择法(SFFS)和Fisher投影法进行特征选择,KNN算法进行分类,获得了81\%的识别准确率\cite{Picard2001},证明了从生理信号中提取特征进行情绪识别研究的可行性。之后,利用生理信号进行情绪识别的研究逐渐发展起来。

实验范式方面,研究中主要采用电影片段情绪诱发、文字/图片情绪诱发、录音/音乐情绪诱发、自传式回忆/想象情绪诱发和具身情境性情绪诱发等典型的情绪诱发方式\cite{Mengzhaolan2005}。
例如,李建平等\cite{LiJianping2006}选用《我的兄弟姐妹》、《午夜凶铃》等六段电影片段诱发92名被试五种基本情绪,发现恐惧、悲伤和中性片段导致HRV高频功率降低,恐惧、愤怒、悲伤及中性片段导致收缩压升高,恐惧、愤怒、厌恶、快乐和中性片段导致呼吸频率加快。
Gomez等\cite{Gomez2004}采用不同正负效价和唤醒度的情绪图片来诱发情绪,发现随着图片效价的增加,被试吸气时间延长,平均吸气流量减少,胸式呼吸增加;随着图片唤醒度的增加,被试吸气时间和总呼吸时间缩短,平均吸气流量、每分钟通气量、胸式呼吸和皮肤电活动增加。
Egloff等\cite{Egloff2002}通过让被试进行演讲的方式诱发其焦虑情绪,发现被试的指端脉搏容积和呼吸频率下降,心率和血压则显著增加。
同时,研究人员也尝试建立标准化的情绪刺激材料数据库,如
美国国立精神卫生研究所建立的标准化国际情绪图片库(IAPS)和英语情感词/短文系统(ANEW/ANET)中国学者\cite{BaiLu2005}建立的本土化中国情绪图片/词库 (CAPS),
IMH建立的国际情感数码声音系统(IADS)和刘涛生等中国学者\cite{LiuTaosheng2006}建立的本土化的中国情感数码声音系统(CADS)等等,这些数据库为统一情绪刺激材料选取范围和标准进行了有效探索。

分析方法方面,研究中主要采用传统的机器学习方法对信号进行特征的提取、选择与融合处理以构建、优化情绪识别模型。
常用的外周生理信号特征\cite{Bota2019}包括时域内的最大/小值、均值/中位数、标准差/方差、平均绝对偏差、过零率、线性回归和积分等,频域内的总能量、光谱质心/扩展/偏度/峰度/斜率/衰减/变化等,非线性中的谱熵、小波熵、样本熵、近似熵、关联维数和Lempel-Ziv复杂度等。
例如,Rubin等\cite{Rubin2016}在恐惧状态的识别中将ECG信号的传统时频域特征和非线性特征相结合,显著提升了模型准确度。
Myroniv等\cite{Myroniv2017}基于RT、DT、NB、KNN、SVM和MPNN模型对HR、GSR和SKT信号进行分类,其中KNN模型的平均准确率达到了最高的97.78\%。
Birjandtalab等\cite{Birjandtalab2016}基于GMM模型对EDA、HR和${\rm SpO}_{\rm 2}$信号进行分类,针对放松、生理紧张、情绪紧张和认知紧张这4种状态均得到了大于84\%的准确率。
越来越多的生理特征被挖掘,为探究生理信号与情绪变化之间的关联建立基础。

除传统机器学习方法外,情绪识别领域也逐渐开始探索和应用深度学习方法(DL),以不依赖于特征的方式进行情绪识别研究。
杨一龙等\cite{YangYilong2018}使用一种结合CNN和RNN的混合神经网络,通过学习原始EEG数据的组成时空表征来区分情绪状态,最终将情绪识别准确率提高了约32\%,在效价分类和唤醒分类上的平均准确率分别为90.80\%和91.03\%。
% 陈景霞等\cite{ChenJX2019}提出一种基于CNN模型的使用EEG特征进行情绪分类的模型,并对比了该模型与决策树、支持向量机、线性判别分析、贝叶斯线性判别分析等在DEAP标准情绪数据集上的识别效果,发现深度学习模型在结合时域与频域的特征集上获得了最好的识别效果。
刘鹏等\cite{刘鹏2021}运用深度栈式自编码(Deep-SAE)方法,从EEG中自动解码并生成特征集合,运用长短时记忆(LSTM)循环网络进行分类模型训练,在DEAP数据集中实现了效价维度77.4\%的情绪识别平均正确率和唤醒度维度73.7\%的平均识别正确率。
利用深度学习进行情绪识别的有效性在大量研究中得到证明,
其识别效果与准确率甚至优于部分传统的机器学习模型,但其缺陷在于解释性很差,不能表征生理信号和情绪之间的对应关系,同时也需要大量的数据和较高的计算成本。

综上,随着基于生理信号进行情绪识别研究的不断推进,研究所聚焦的情绪种类,选用的生理信号,提取的生理特征以及分类模型的构建和优化方法越来越丰富和全面,
在提升情绪分类效果,完善情绪诱发方案和构建标准情绪数据库等多个方面取得了很多成果,但仍存在以下三个问题:

(1) 从生理信号上看,现有研究多基于脑电信号(EEG),但EEG的实验采集过程(包括操作、设备等)相对复杂和困难,使得研究难以适用于长时间的情绪监测以及情绪识别进入日常生活的应用趋势。

(2) 从特征维度上看,现有研究提取的情绪特征较为单一,多为时域、频域和非线性特征中的一种或几种,导致情绪特征的正交性和分类模型的全面性不足。

(3) 从情绪与生理信号的关联上看,现有研究聚焦于提升情绪识别准确率,即主要关注生理信号的种类选择、特征的提取与筛选、模型的构建与优化等方面,而缺乏对生理信号与情绪状态之间的特异性关联的研究,对情绪状态的改变如何影响以及何种程度影响生理信号表征等缺乏成熟的理论进行详细阐释。

\subsection{本研究的意义和目的}
在常见的负性情绪中,恐惧是一种影响较深的高唤醒度状态\cite{James1884},是人类自身对于威胁的有效预警,根本目的在于保护机体免受外界伤害。
但长期处于恐惧状态会催生和积累更多的负面情绪(如愤怒、悲伤等),进而导致人脑功能不同程度的损伤,使人的知觉、记忆、分析和判断等过程发生障碍,严重时甚至会导致生理功能的紊乱和行为能力的丧失。
而在面对恐惧等消极情绪完全超出控制的情况下,愉悦等积极情绪可以有效缓解个体的心理压力及其机体的消极反应,有效弱化负面影响。因此对恐惧和愉悦情绪的研究对于稳定情绪状态和改善生活质量有很大的实际意义。
论文希望融合多种外周生理信号,提取多个维度的特征参数对恐惧和愉悦情绪以及平静状态进行识别,更加全面和深入地研究相关外周生理特征与两种基本情绪之间的关联,以达到更好的识别效果。
同时,研究聚焦于外周生理信号,有利于后续情绪识别研究在实验设备小型化、便携化上的进一步拓展,有利于后续进一步研究情绪识别在日常生活中的应用。

\subsection{主要研究内容}
根据心理学研究中常用的情绪分类模型和情绪生理机制,本研究主要聚焦的情绪种类为负性情绪-恐惧、正性情绪-愉悦和中性情绪-平静,相应采集的外周生理信号包括心电、脉搏波和呼吸信号,
旨在通过以上三种外周生理信号实现对部分基本情绪的识别。研究的主要内容包括以下五个部分:

(1)情绪刺激材料评价实验的方案设计与数据分析;

(2)情绪生理唤醒实验的方案设计与生理信号的采集;

(3)三种外周生理信号的预处理与特征提取;

(4)生理特征的统计学分析;

(5)情绪分类模型的构建、优化与效果评估。